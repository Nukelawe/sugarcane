\documentclass[12pt]{article}
\usepackage[english]{babel}
\usepackage{t1enc}

% AMS packages:
\usepackage{amsbsy}
\usepackage{amsfonts}
\usepackage{amsmath}
\usepackage{amssymb}
\usepackage{amsthm}
\usepackage{amsxtra}
% for comments to work
\usepackage{verbatim}
\usepackage[colorlinks=true,linkcolor=blue,urlcolor=blue,citecolor=blue,anchorcolor=blue]{hyperref}
% other packages
\usepackage{graphicx}
\usepackage{mathtools}
\usepackage{cancel} % term cancellations in math mode
\usepackage{units}
\usepackage{caption} % For more control over figure captions
\usepackage{subcaption}
\usepackage[section]{placeins}
\usepackage{datatool}
\usepackage{titlesec}
\usepackage{titling}
\usepackage{enumitem}
\usepackage{xcolor}
\usepackage{fancyhdr}
\usepackage{tikz} % for drawing vector graphics
\usepackage{tikz-3dplot}
%\counterwithin{figure}{section}
\usetikzlibrary{positioning}
\usetikzlibrary{arrows.meta}
\captionsetup{width=.9\linewidth}
\captionsetup{labelfont=bf}
\captionsetup{font=small}

%% use an external data file
%\DTLsetseparator{ }
%\DTLloaddb[noheader, keys={key,value}]{database}{p4.dat}
%% macro for reading a value from database
%\newcommand{\dataread}[1]{\DTLfetch{database}{key}{#1}{value}}

% \pagenumbering{gobble} % remove page numbers

\hoffset 0.5cm
\voffset -0.4cm
\evensidemargin -0.2in
\oddsidemargin -0.2in
\topmargin -0.2in
\textwidth 6.3in
\textheight 8.4in

\setlength{\parindent}{0em}
\setlength{\parskip}{.5em}

%%%%%%%%%%%%%%%%%%%%%%%%%%%%%%%%%%%
% Macro definitions
%%%%%%%%%%%%%%%%%%%%%%%%%%%%%%%%%%%

% operator with description
\newcommand{\op}[2]{\stackrel{\text{\scriptsize #2}}{#1}}
% indicator function
\newcommand{\ind}[1]{\mathbb{I}_{#1}}
\newcommand{\id}[1]{\mathbbm{1}\!\left({#1}\right)}
% probability and statistics
\newcommand{\Prob}[1]{\begingroup
		\renewcommand\mid{\,|\,}
		\text{$\mathbb{P}\left(#1\right)$}
	\endgroup
}
\newcommand{\Exp}[1]{\begingroup
		\renewcommand\mid{\,|\,}
		\text{$\mathbb{E}\left[#1\right]$}
	\endgroup
}
% Hilbert spaces:
\newcommand{\hilb}{\mathcal{H}}
\newcommand{\banH}{\mathcal{B}(\hilb)}
\newcommand{\fock}{\mathcal{F}}
% Products:
\newcommand{\scalpr}[2]{( #1, #2 )}
\newcommand{\dualpr}[2]{\langle #1, #2 \rangle}
% Real and imaginary parts
\newcommand{\re}{{\rm Re\,}}
\newcommand{\im}{{\rm Im\,}}
% misc
\newcommand{\defem}[1]{{\em #1\/}}
% To define sets:
\newcommand{\defset}[2]{
	\left\{ #1 :\, #2\makebox[0pt]{$\displaystyle\phantom{#1}$}\right\}}

% Norms:
\newcommand{\abs}[1] {\lvert #1 \rvert}
\newcommand{\norm}[1]{\left\lVert #1 \right\rVert}
\newcommand{\floor}[1] {\lfloor {#1} \rfloor}
\newcommand{\ceil}[1]  {\lceil  {#1} \rceil}

% Basic spaces
\newcommand{\R} {\mathbb{R}}
\newcommand{\C} {{\mathbb{C}}}
\newcommand{\N} {\mathbb{N}}
\newcommand{\Z} {\mathbb{Z}}
\newcommand{\Q} {\mathbb{Q}}
\newcommand{\K} {\mathbb{K}}
\newcommand{\T} {\mathbb{T}}

% differentials
\newcommand{\dd}{\,\mathrm{d}}
\newcommand{\DD}{\partial}
% partial derivatives
\newcommand{\pder}[2]{\frac{\DD #1}{\DD #2}}
\newcommand{\pdersq}[2]{\frac{\DD^2 #1}{\DD #2^2}}
\newcommand{\pderev}[3]{\left.\pder{#1}{#2}\right|_{#3}}
% total derivatives
\newcommand{\der}[2]{\frac{\dd #1}{\dd #2}}
\newcommand{\derev}[3]{\left.\der{#1}{#2}\right|_{#3}}
% Fonts
\newcommand{\Acal}{\mathcal{A}}
\newcommand{\Bcal}{\mathcal{B}}
\newcommand{\Ccal}{\mathcal{C}}
\newcommand{\Dcal}{\mathcal{D}}
\newcommand{\Fcal}{\mathcal{F}}
\newcommand{\Hcal}{\mathcal{H}}
\newcommand{\Lcal}{\mathcal{L}}
\newcommand{\Ocal}{\mathcal{O}}
\newcommand{\Pcal}{\mathcal{P}}
\newcommand{\Rcal}{\mathcal{R}}
\newcommand{\Scal}{\mathcal{S}}
\newcommand{\Tcal}{\mathcal{T}}
\newcommand{\Ucal}{\mathcal{U}}
\newcommand{\Vcal}{\mathcal{V}}
\newcommand{\abf}{{\bf a}}
\newcommand{\Zcal}{{\cal Z}_{12}}
\newcommand{\zcal}{z_{12}}
\newcommand{\hsf}{{\sf h}}
\newcommand{\half}{\frac{1}{2}}
\newcommand{\Xbar}{\bar{X}}
\newcommand{\xibar}{\bar{\xi }}
\newcommand{\barh}{\bar{h}}
\newcommand{\Ubar}{\bar{\cal U}}
\newcommand{\Vbar}{\bar{\cal V}}
\newcommand{\Fbar}{\bar{F}}
\newcommand{\zbar}{\bar{z}}
\newcommand{\wbar}{\bar{w}}
\newcommand{\wbartilde}{\tilde{\bar{w}}}
\newcommand{\barone}{\bar{1}}
\newcommand{\bartwo}{\bar{2}}
\newcommand{\nbyn}{N \times N}
\newcommand{\repres}{\leftrightarrow}
\newcommand{\Tr}{{\rm Tr}}
\newcommand{\tr}{{\rm tr}}
\newcommand{\ninfty}{N \rightarrow \infty}
\newcommand{\unitk}{{\bf 1}_k}
\newcommand{\unitm}{{\bf 1}}
\newcommand{\zerom}{{\bf 0}}
\newcommand{\unittwo}{{\bf 1}_2}
\newcommand{\holo}{{\cal U}}
\newcommand{\bra}{\langle}
\newcommand{\ket}{\rangle}
\newcommand{\sgamma}{\sqrt{\gamma}}
\newcommand{\bfE}{{\bf E}}
\newcommand{\bfB}{{\bf B}}
\newcommand{\bfM}{{\bf M}}
\newcommand{\bfT}{{\bf T}}
\newcommand{\cl} {\cal l}
\newcommand{\ctilde}{\tilde{\chi}}
\newcommand{\ttilde}{\tilde{t}}
\newcommand{\ptilde}{\tilde{\phi}}
\newcommand{\utilde}{\tilde{u}}
\newcommand{\vtilde}{\tilde{v}}
\newcommand{\wtilde}{\tilde{w}}
\newcommand{\ztilde}{\tilde{z}}

\selectlanguage{english}


\begin{document}
	\title{Sugar cane harvesting period optimization in Minecraft}
	\author{Nukelawe}
	\maketitle
	\section{Theory}
	Consider a sugar cane plant that is harvested periodically, every $t$ game ticks.
	The state of the sugar cane is identified by its height ($\in\{1,2,3\}$) and the
	internal age value of the top-most block ($\in\{0,\ldots,15\}$).

	When the sugar cane with height less than 3 receives a random tick its age is
	incremented by one. If the age is already 15 then it is reset to 0 and the height is
	incremented by one. On every random tick attempt (of which there are 3 per tick by
	default) the probability that the sugar cane
	receives a random tick is $q=\frac{1}{4096}$.

	When a sugar cane is harvested its height is set to 1. The age value is reset to 0
	unless the height of the sugar cane was already 1 before the harvest in which case the
	age will be unmodified. We assume that after the harvest the sugar cane is blocked by
	the harvesting machine for $s$ ticks. This implies that during the first $s$ ticks
	after the harvest any attempts to increase the age or height of the sugar cane will
	fail, even age changes that wouldn't cause the sugar cane to increase in height.

	Let $H_n$ be the height of the sugar cane plant $n$ ticks after a harvest. Because
	harvesting removes all but the lowest sugar cane we have $H_0=1$.
	Let $R_n$ be the number of random ticks received by the plant in the last $n$ ticks.
	It is binomially distributed with $3n$ trials and success probability $q$, i.e. $R_n
	\sim B(3n,q)$.
	At the end of the harvest cycle the sugar cane height distribution is given by
	\begin{align}\label{eq:heightDistribution}
		\Prob{H_t=i} &= \begin{cases}
			\Prob{R_{t-s}+X < 16} &\text{if }i=1
			\\\Prob{16 \leq R_{t-s}+X < 32} \quad&\text{if }i=2
			\\\Prob{32 \leq R_{t-s}+X} &\text{if }i=3
		\end{cases}
	\end{align}
	where $X$ is the internal age value of the sugar cane block immediately after the
	harvest. The reason why $R_{t-s}$ appears in (\ref{eq:heightDistribution}) instead of
	$R_t$ is that for the first $s$ ticks of the harvest cycle the sugar cane is assumed
	to be blocked by the harvesting machine.

	We would now like to optimize the expected harvest rate
	\begin{align}\label{eq:harvestRate}
		v = \frac{\langle H_t-1\rangle}{t}
		= \frac{\Prob{H_t=2} + 2\Prob{H_t=3}}{t}
	\end{align}
	by choosing a suitable harvest period $t$. However, to compute the probabilities in
	(\ref{eq:harvestRate}) we must know the initial age parameter $X$, which depends on
	the previous harvest. With this in mind let us define $X_m$ to be the age of the sugar
	cane immediately after the $m$th harvest. The sequence $X_0,X_1,\ldots$ is a Markov
	process as the age parameter after $m+1$ harvests $X_{m+1}$ only depends on the one
	after $m$ harvests, $X_m$.

	The relation between $X_m$ and $X_{m+1}$ is characterized by the transition
	probabilities
	\begin{align}\label{eq:transitionProbability}
		p_{i,j} := \Prob{X_{m+1}\!=\!j \mid X_m\!=\!i}
		= \begin{cases}
			0 \quad&\text{if }0<j<i
			\\\Prob{R_{t-s}\!=\!j-i} \quad&\text{if }j\geq i
			\\\Prob{R_{t-s}\!\geq\!16-i} \quad&\text{if }j=0,i>0
			\\\Prob{R_{t-s}\!\geq\!16}+\Prob{R_{t-s}\!=\!0} \quad&\text{if }j=0,i=0
		\end{cases}
	\end{align}
	and since we know that $R_{t-s}\sim B(3(t-s),q)$ the probabilities that appear have the
	following analytic expressions.
	\begin{align}
		\Prob{R_{t-s}\!=\!j-i}\label{eq:binomialCdf}
		&= {t-s\choose j-i}q^{j-i}(1-q)^{t-s-j+i}
		\\\Prob{R_{t-s}\!\geq\!16-i}\label{eq:binomialCdf}
		&= 1 - \sum_{k=0}^{15-i}{t-s\choose k-i}q^{k-i}(1-q)^{t-s-k+i}
	\end{align}

	Define the 16-dimensional probability vector
	$\mathbf{P}_m = \big(\Prob{X_m\!=\!0}, \Prob{X_m\!=\!1}, \cdots,
	\Prob{X_m\!=\!15}\big)$
	to represent the age distribution after the $m$th harvest and the 16-by-16 transition
	matrix
	\begin{align}
		\mathbf{T}
		&= \left(\begin{matrix}
			p_{0,0} & p_{0,1} & \cdots & p_{0,15}
			\\p_{1,0} & p_{1,1} & \cdots & p_{1,15}
			\\\vdots & \vdots & \ddots & \vdots
			\\p_{15,0} & p_{15,1} & \cdots & p_{15,15}
		\end{matrix}\right)
	\end{align}
	to tabulate the transition probabilities.
	Now $\mathbf{P}_{m+1} = \mathbf{P}_m\mathbf{T}$ and $\mathbf{P}_m =
	\mathbf{P}_0\mathbf{T}^m$. It is probably fair to assume that before the first
	harvest ($m=0$) the age state of the sugar cane is 0. This could be because either the
	sugar can was just planted or the harvesting machine had been turned off for so long
	that the sugar cane had grown past height 1. Therefore, we set $\mathbf{P}_0
	= \big(1,0,0,\ldots,0\big)$, i.e. $X_0=0$ with probability 1.
	Over many harvest cycles the initial age state of the sugar cane will converge on a
	stationary distribution $\mathbf{P}_\infty =
	\lim\limits_{m\rightarrow\infty}\mathbf{P}_m$,
	which satisfies
	\begin{align}
		\mathbf{P}_\infty = \mathbf{P}_\infty\mathbf{T} \iff
		\mathbf{P}_\infty(\mathbf{T}-\mathbf{1})=\mathbf{0}.
	\end{align}
	In other words, $\mathbf{P}_\infty$ is an eigen vector of the transition matrix.

	When determining the optimal harvest period it makes sense to use the steady state
	distribution as the initial sugar cane age distribution as this what we would see
	after operating the sugar cane farm for a longer time.
	With this we can finally compute
	the sugar cane height distribution at the time of harvest starting from
	(\ref{eq:heightDistribution}). Computing the harvest rate (\ref{eq:harvestRate})
	only requires us to know the probabilities for sugar cane heights 2 and 3.
	\begin{align}\label{eq:heightDistribution2}
		\Prob{H_t=2} &=
			\sum_{k=0}^{15}\Prob{X_\infty\!=\!k}\,\Prob{16-k\!\leq\!R_{t-s}\!<\!32-k}
		\\\Prob{H_t=3} &=
			\sum_{k=0}^{15}\Prob{X_\infty\!=\!k}\,\Prob{R_{t-s}\!\geq\!32-k}
	\end{align}
	where $X_\infty$ is the random variable that represents the age state of the sugar
	cane at the beginning of the $m$th harvest for $m\rightarrow\infty$.

	\section{Results}
	One might ask how quickly the initial age distribution converges on the stationary
	solution, how many harvests cycles should we wait until the use of the stationary
	solution in the harvest rate calculations is justified?
	Figure~\ref{fig:harvestRateDistributionComparison} illustrates how the initial age
	distribution affects the harvest rate. We assume that there is when the harvesting
	machine is turned on the sugar cane is immediately broken and its age state is set to
	0. Therefore the age distribution at the beginning of the first harvest cycle is
	assumed to be given by $\mathbf{P}_0 = (1,0,\ldots,0)$. Looking at the curve labeled
	``1. harvest" we notice that to maximize the
	harvest rate for the first harvest cycle, the harvest period should be about 41
	minutes, which coincides with a \emph{local} maximum of the stationary case. To
	maximize the rate for the \emph{second} harvest rate however, we would have to set the
	harvest period to about 11 minutes. The location of the global maximum keeps
	increasing as more and more harvest cycles are considered until in the stationary case
	the maximum is obtained at $t=0$.
	\begin{figure}[h!]
		\centering
		\includegraphics{sugarcane.pdf}
		\caption{The harvest rate $v$ (see (\ref{eq:harvestRate})) graphed for different
		harvest periods $t$. Here it was assumed that the harvesting machine does
		\emph{not} block plant growth at all, i.e. $s=0$. The different curves represent
		harvest rates depending on the number of harvest cycles since the sugar cane was
		planted or the harvesting machine was turned on. The peak of the
		1\textsuperscript{st} harvest rate curve, denoted by the red vertical line, occurs
		at $t=49252\,$ticks or about 41 minutes.}
		\label{fig:harvestRateDistributionComparison}
	\end{figure}

	This begs the question how much better it would be to use an adaptive harvest period
	that changes based on the number of harvest cycles since turning the harvesting
	machine on. In particular, we would set the delay between the $(m-1)$th and $m$th
	harvests to value at which the harvest rate for the $m$th harvest is maximized.
	However, because the global maximum of the 1\textsuperscript{st} harvest rate curve
	occurs at the same point as that of the stationary curve we conclude that such an
	adaptive strategy would just result in regular harvest periods of about 41 minutes,
	which would not be adaptive at all.

	Figure~\ref{fig:harvestRateDistributionComparison} shows that for the first couple of
	harvest cycles choosing a small ($<10$ minutes) harvest period would not be very
	productive at all, but eventually as more harvest cycles are considered a small period
	will actually result in higher rates than the 41 minute strategy. It would be
	interesting to know how many harvest cycles must be considered before using a small
	harvest period becomes more efficient. To figure this out we compute the expected
	harvest rate over the first $m$ harvest cycles. Figure~\ref{fig:sugarcane2} shows how
	long a harvesting machine must be operated until it starts to perform better than the
	41 minute strategy. Even though it might take hundreds of harvest cycles to reach this
	point, the total time is not that long since the harvest cycles also become quite
	short.
	\begin{figure}[h!]
		\centering
		\includegraphics{sugarcane2.pdf}
		\caption{Graph showing how long a sugar cane farm should be run with a small clock
		speed to reach at least as good a rate as by harvesting every 41 ticks.}
		\label{fig:sugarcane2}
	\end{figure}

	So far we have only considered no growth blocking by the harvesting machine ($s=0$).
	Figure~\ref{fig:sugarcane3} shows how the harvest rates (assuming stationary age
	distribution) are affected by growth blocking. Unsurprisingly, this has very little
	effect on the long harvest periods. Now the global maximum is attained at some
	non-zero value of the harvest period since for $t\leq s$ growth is prevented all the
	time and therefore the harvest rate is 0.
	\begin{figure}[h!]
		\centering
		\includegraphics{sugarcane3.pdf}
		\caption{Effect of growth blocking on the harvest rates. Only smaller harvest
		periods are shown.}
		\label{fig:sugarcane3}
	\end{figure}
\end{document}
